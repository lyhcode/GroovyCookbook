% Generated by Sphinx.
\def\sphinxdocclass{report}
\documentclass[a4paper,12pt,english]{sphinxmanual}
\usepackage[utf8]{inputenc}
\DeclareUnicodeCharacter{00A0}{\nobreakspace}
\usepackage[T1]{fontenc}
\usepackage{babel}
\usepackage{times}
\usepackage[Bjarne]{fncychap}
\usepackage{longtable}
\usepackage{sphinx}
\usepackage[cm-default]{fontspec}
\usepackage{xunicode}
\usepackage{xcolor}
\usepackage{fontspec}
\usepackage{titlesec}
\usepackage{fancyvrb,relsize}
\usepackage[slantfont,boldfont]{xeCJK}
\XeTeXlinebreaklocale "zh"
\XeTeXlinebreakskip = 0pt plus 1pt
\setmainfont[BoldFont=Apple LiGothic Medium]{Apple LiSung Light}
\setCJKmainfont[BoldFont=Apple LiGothic Medium]{Apple LiSung Light}
\setromanfont[BoldFont=Apple LiGothic Medium]{Apple LiSung Light}
\setmonofont{Monaco}
\renewcommand{\baselinestretch}{1.25}
\DefineVerbatimEnvironment{Verbatim}{Verbatim}{numbers=left, fontsize=\relsize{-1}}


\title{Groovy Cookbook}
\date{October 04, 2011}
\release{1.0}
\author{by lyhcode}
\newcommand{\sphinxlogo}{\includegraphics{cover.eps}\par}
\renewcommand{\releasename}{Release}
\makeindex

\makeatletter
\def\PYG@reset{\let\PYG@it=\relax \let\PYG@bf=\relax%
    \let\PYG@ul=\relax \let\PYG@tc=\relax%
    \let\PYG@bc=\relax \let\PYG@ff=\relax}
\def\PYG@tok#1{\csname PYG@tok@#1\endcsname}
\def\PYG@toks#1+{\ifx\relax#1\empty\else%
    \PYG@tok{#1}\expandafter\PYG@toks\fi}
\def\PYG@do#1{\PYG@bc{\PYG@tc{\PYG@ul{%
    \PYG@it{\PYG@bf{\PYG@ff{#1}}}}}}}
\def\PYG#1#2{\PYG@reset\PYG@toks#1+\relax+\PYG@do{#2}}

\expandafter\def\csname PYG@tok@gd\endcsname{\def\PYG@tc##1{\textcolor[rgb]{0.63,0.00,0.00}{##1}}}
\expandafter\def\csname PYG@tok@gu\endcsname{\let\PYG@bf=\textbf\def\PYG@tc##1{\textcolor[rgb]{0.50,0.00,0.50}{##1}}}
\expandafter\def\csname PYG@tok@gt\endcsname{\def\PYG@tc##1{\textcolor[rgb]{0.00,0.25,0.82}{##1}}}
\expandafter\def\csname PYG@tok@gs\endcsname{\let\PYG@bf=\textbf}
\expandafter\def\csname PYG@tok@gr\endcsname{\def\PYG@tc##1{\textcolor[rgb]{1.00,0.00,0.00}{##1}}}
\expandafter\def\csname PYG@tok@cm\endcsname{\let\PYG@it=\textit\def\PYG@tc##1{\textcolor[rgb]{0.25,0.50,0.56}{##1}}}
\expandafter\def\csname PYG@tok@vg\endcsname{\def\PYG@tc##1{\textcolor[rgb]{0.73,0.38,0.84}{##1}}}
\expandafter\def\csname PYG@tok@m\endcsname{\def\PYG@tc##1{\textcolor[rgb]{0.13,0.50,0.31}{##1}}}
\expandafter\def\csname PYG@tok@mh\endcsname{\def\PYG@tc##1{\textcolor[rgb]{0.13,0.50,0.31}{##1}}}
\expandafter\def\csname PYG@tok@cs\endcsname{\def\PYG@tc##1{\textcolor[rgb]{0.25,0.50,0.56}{##1}}\def\PYG@bc##1{\setlength{\fboxsep}{0pt}\colorbox[rgb]{1.00,0.94,0.94}{\strut ##1}}}
\expandafter\def\csname PYG@tok@ge\endcsname{\let\PYG@it=\textit}
\expandafter\def\csname PYG@tok@vc\endcsname{\def\PYG@tc##1{\textcolor[rgb]{0.73,0.38,0.84}{##1}}}
\expandafter\def\csname PYG@tok@il\endcsname{\def\PYG@tc##1{\textcolor[rgb]{0.13,0.50,0.31}{##1}}}
\expandafter\def\csname PYG@tok@go\endcsname{\def\PYG@tc##1{\textcolor[rgb]{0.19,0.19,0.19}{##1}}}
\expandafter\def\csname PYG@tok@cp\endcsname{\def\PYG@tc##1{\textcolor[rgb]{0.00,0.44,0.13}{##1}}}
\expandafter\def\csname PYG@tok@gi\endcsname{\def\PYG@tc##1{\textcolor[rgb]{0.00,0.63,0.00}{##1}}}
\expandafter\def\csname PYG@tok@gh\endcsname{\let\PYG@bf=\textbf\def\PYG@tc##1{\textcolor[rgb]{0.00,0.00,0.50}{##1}}}
\expandafter\def\csname PYG@tok@ni\endcsname{\let\PYG@bf=\textbf\def\PYG@tc##1{\textcolor[rgb]{0.84,0.33,0.22}{##1}}}
\expandafter\def\csname PYG@tok@nl\endcsname{\let\PYG@bf=\textbf\def\PYG@tc##1{\textcolor[rgb]{0.00,0.13,0.44}{##1}}}
\expandafter\def\csname PYG@tok@nn\endcsname{\let\PYG@bf=\textbf\def\PYG@tc##1{\textcolor[rgb]{0.05,0.52,0.71}{##1}}}
\expandafter\def\csname PYG@tok@no\endcsname{\def\PYG@tc##1{\textcolor[rgb]{0.38,0.68,0.84}{##1}}}
\expandafter\def\csname PYG@tok@na\endcsname{\def\PYG@tc##1{\textcolor[rgb]{0.25,0.44,0.63}{##1}}}
\expandafter\def\csname PYG@tok@nb\endcsname{\def\PYG@tc##1{\textcolor[rgb]{0.00,0.44,0.13}{##1}}}
\expandafter\def\csname PYG@tok@nc\endcsname{\let\PYG@bf=\textbf\def\PYG@tc##1{\textcolor[rgb]{0.05,0.52,0.71}{##1}}}
\expandafter\def\csname PYG@tok@nd\endcsname{\let\PYG@bf=\textbf\def\PYG@tc##1{\textcolor[rgb]{0.33,0.33,0.33}{##1}}}
\expandafter\def\csname PYG@tok@ne\endcsname{\def\PYG@tc##1{\textcolor[rgb]{0.00,0.44,0.13}{##1}}}
\expandafter\def\csname PYG@tok@nf\endcsname{\def\PYG@tc##1{\textcolor[rgb]{0.02,0.16,0.49}{##1}}}
\expandafter\def\csname PYG@tok@si\endcsname{\let\PYG@it=\textit\def\PYG@tc##1{\textcolor[rgb]{0.44,0.63,0.82}{##1}}}
\expandafter\def\csname PYG@tok@s2\endcsname{\def\PYG@tc##1{\textcolor[rgb]{0.25,0.44,0.63}{##1}}}
\expandafter\def\csname PYG@tok@vi\endcsname{\def\PYG@tc##1{\textcolor[rgb]{0.73,0.38,0.84}{##1}}}
\expandafter\def\csname PYG@tok@nt\endcsname{\let\PYG@bf=\textbf\def\PYG@tc##1{\textcolor[rgb]{0.02,0.16,0.45}{##1}}}
\expandafter\def\csname PYG@tok@nv\endcsname{\def\PYG@tc##1{\textcolor[rgb]{0.73,0.38,0.84}{##1}}}
\expandafter\def\csname PYG@tok@s1\endcsname{\def\PYG@tc##1{\textcolor[rgb]{0.25,0.44,0.63}{##1}}}
\expandafter\def\csname PYG@tok@gp\endcsname{\let\PYG@bf=\textbf\def\PYG@tc##1{\textcolor[rgb]{0.78,0.36,0.04}{##1}}}
\expandafter\def\csname PYG@tok@sh\endcsname{\def\PYG@tc##1{\textcolor[rgb]{0.25,0.44,0.63}{##1}}}
\expandafter\def\csname PYG@tok@ow\endcsname{\let\PYG@bf=\textbf\def\PYG@tc##1{\textcolor[rgb]{0.00,0.44,0.13}{##1}}}
\expandafter\def\csname PYG@tok@sx\endcsname{\def\PYG@tc##1{\textcolor[rgb]{0.78,0.36,0.04}{##1}}}
\expandafter\def\csname PYG@tok@bp\endcsname{\def\PYG@tc##1{\textcolor[rgb]{0.00,0.44,0.13}{##1}}}
\expandafter\def\csname PYG@tok@c1\endcsname{\let\PYG@it=\textit\def\PYG@tc##1{\textcolor[rgb]{0.25,0.50,0.56}{##1}}}
\expandafter\def\csname PYG@tok@kc\endcsname{\let\PYG@bf=\textbf\def\PYG@tc##1{\textcolor[rgb]{0.00,0.44,0.13}{##1}}}
\expandafter\def\csname PYG@tok@c\endcsname{\let\PYG@it=\textit\def\PYG@tc##1{\textcolor[rgb]{0.25,0.50,0.56}{##1}}}
\expandafter\def\csname PYG@tok@mf\endcsname{\def\PYG@tc##1{\textcolor[rgb]{0.13,0.50,0.31}{##1}}}
\expandafter\def\csname PYG@tok@err\endcsname{\def\PYG@bc##1{\setlength{\fboxsep}{0pt}\fcolorbox[rgb]{1.00,0.00,0.00}{1,1,1}{\strut ##1}}}
\expandafter\def\csname PYG@tok@kd\endcsname{\let\PYG@bf=\textbf\def\PYG@tc##1{\textcolor[rgb]{0.00,0.44,0.13}{##1}}}
\expandafter\def\csname PYG@tok@ss\endcsname{\def\PYG@tc##1{\textcolor[rgb]{0.32,0.47,0.09}{##1}}}
\expandafter\def\csname PYG@tok@sr\endcsname{\def\PYG@tc##1{\textcolor[rgb]{0.14,0.33,0.53}{##1}}}
\expandafter\def\csname PYG@tok@mo\endcsname{\def\PYG@tc##1{\textcolor[rgb]{0.13,0.50,0.31}{##1}}}
\expandafter\def\csname PYG@tok@mi\endcsname{\def\PYG@tc##1{\textcolor[rgb]{0.13,0.50,0.31}{##1}}}
\expandafter\def\csname PYG@tok@kn\endcsname{\let\PYG@bf=\textbf\def\PYG@tc##1{\textcolor[rgb]{0.00,0.44,0.13}{##1}}}
\expandafter\def\csname PYG@tok@o\endcsname{\def\PYG@tc##1{\textcolor[rgb]{0.40,0.40,0.40}{##1}}}
\expandafter\def\csname PYG@tok@kr\endcsname{\let\PYG@bf=\textbf\def\PYG@tc##1{\textcolor[rgb]{0.00,0.44,0.13}{##1}}}
\expandafter\def\csname PYG@tok@s\endcsname{\def\PYG@tc##1{\textcolor[rgb]{0.25,0.44,0.63}{##1}}}
\expandafter\def\csname PYG@tok@kp\endcsname{\def\PYG@tc##1{\textcolor[rgb]{0.00,0.44,0.13}{##1}}}
\expandafter\def\csname PYG@tok@w\endcsname{\def\PYG@tc##1{\textcolor[rgb]{0.73,0.73,0.73}{##1}}}
\expandafter\def\csname PYG@tok@kt\endcsname{\def\PYG@tc##1{\textcolor[rgb]{0.56,0.13,0.00}{##1}}}
\expandafter\def\csname PYG@tok@sc\endcsname{\def\PYG@tc##1{\textcolor[rgb]{0.25,0.44,0.63}{##1}}}
\expandafter\def\csname PYG@tok@sb\endcsname{\def\PYG@tc##1{\textcolor[rgb]{0.25,0.44,0.63}{##1}}}
\expandafter\def\csname PYG@tok@k\endcsname{\let\PYG@bf=\textbf\def\PYG@tc##1{\textcolor[rgb]{0.00,0.44,0.13}{##1}}}
\expandafter\def\csname PYG@tok@se\endcsname{\let\PYG@bf=\textbf\def\PYG@tc##1{\textcolor[rgb]{0.25,0.44,0.63}{##1}}}
\expandafter\def\csname PYG@tok@sd\endcsname{\let\PYG@it=\textit\def\PYG@tc##1{\textcolor[rgb]{0.25,0.44,0.63}{##1}}}

\def\PYGZbs{\char`\\}
\def\PYGZus{\char`\_}
\def\PYGZob{\char`\{}
\def\PYGZcb{\char`\}}
\def\PYGZca{\char`\^}
\def\PYGZam{\char`\&}
\def\PYGZlt{\char`\<}
\def\PYGZgt{\char`\>}
\def\PYGZsh{\char`\#}
\def\PYGZpc{\char`\%}
\def\PYGZdl{\char`\$}
\def\PYGZti{\char`\~}
% for compatibility with earlier versions
\def\PYGZat{@}
\def\PYGZlb{[}
\def\PYGZrb{]}
\makeatother

\begin{document}

\maketitle
\tableofcontents
\phantomsection\label{index::doc}



\chapter{簡介}
\label{intro:groovy-cookbook}\label{intro::doc}\label{intro:id1}

\section{認識 Groovy}
\label{intro:groovy}
面對時下許多 Scripting 程式語言的競爭,
如 PHP / Python / Perl / Ruby 等,
Java 程式寫起來總是充滿「麻煩與複雜」。
可是 Java 跨平台的特性、龐大豐富的函式庫,
卻又令許多開發者無法割捨。

對於已經熟悉 Java 的程式設計師,
未來該何去何從呢?

是否要繼續耗費更多倍的心力,
只為達到其它程式語言輕鬆快速就能實現的功能?
或者乾脆拋棄舊愛,
投向其它程式語言的懷抱?

作者從事 Java 軟體開發的工作已有五年光陰,
從一開始不喜歡 Java,
到後來非常討厭 Java,
卻仍然無法向 Java 說掰掰!
手上的許多重要專案,
主要的架構仍是採用 Java 開發。

有兩個主要因素導致這樣的結果:
\begin{enumerate}
\item {} 
跨平台、跨程式語言的整合已經相當容易,
用什麼程式語言開發其實不重要。
選擇「合適的程式語言」解決特定的問題才是重點。
Java 既有諸多成熟豐富的函式庫,
仍是目前很不錯的解決方案。

\item {} 
在 Java 的世界中,
早已有許多新興的 JVM Scripting Lanaguage,
可以在既有的 JVM 平台上執行,
並且與原來的 Java 程式無痛整合、無縫銜接。

\end{enumerate}

目前較廣泛採用的 JVM Scripting Language,
包括 Jython 、 JRuby 、 Groovy 及 Scala 等。

本書要介紹的 Groovy ,
算是 Java 的「方言」,
它的語法與 Java 最為接近,
但是大幅簡化。
對 Java 程式設計師而言,
Groovy 的學習曲線最為平滑,
而且投資報酬率高,
認識愈多 Groovy 的精簡寫法,
就會得到愈高的 Java 程式生產力,
獲得事半功倍的成效。


\section{安裝 Groovy}
\label{intro:id2}
只要成功安裝 Java SDK 的作業系統,
通常就能使用 Groovy 開發環境,
例如:
\begin{itemize}
\item {} 
Windows

\item {} 
Linux

\item {} 
Mac OS X

\end{itemize}


\subsection{Ubuntu Linux}
\label{intro:ubuntu-linux}
Ubuntu 的套件庫已提供 Groovy,
可以使用 apt-get 指令安裝。

\begin{Verbatim}[commandchars=@\[\]]
sudo apt-get update
sudo apt-get install groovy
\end{Verbatim}

但是 Groovy 目前的版本更新速度相當快,
Ubuntu 套件庫提供的 Groovy 通常是比較舊的版本,
建議從 Launchpad PPA 安裝最新版本。

\begin{Verbatim}[commandchars=@\[\]]
sudo apt-add-repository ppa:groovy-dev/groovy
sudo apt-get update
sudo apt-get install groovy-1.8
\end{Verbatim}


\section{執行 Groovy 程式}
\label{intro:id3}

\subsection{GroovyConsole}
\label{intro:groovyconsole}
對剛開始接觸 Groovy 程式語言的學習者,
GroovyConsole 可能是一個最友善的工具,
它提供圖形化介面及程式碼編輯功能,
並可以在按 Executive 按鈕後立即執行程式碼。

請執行 groovyConsole 程式,並依照以下步驟使用。
\begin{enumerate}
\item {} 
輸入一段程式碼
(1..10).each \{println it\}

\item {} 
按上方選單的 Script \textgreater{} Run 執行

\end{enumerate}

執行成功將會看到下方的訊息輸出區,依序顯示 1 至 10 的數字。

以下是 GroovyConsole 常用的快速鍵。

\begin{tabulary}{\linewidth}{|L|L|L|}
\hline
\textbf{
用途
} & \textbf{
Windows \& Linux
} & \textbf{
Mac OS X
}\\
\hline

執行
 & 
Ctrl + R
 & 
Command + R
\\

清除輸出
 & 
Ctrl + W
 & 
Command + W
\\
\hline
\end{tabulary}



\subsection{GroovyShell}
\label{intro:groovyshell}
Groovy 也提供一種交談式的 Shell 程式,
每一次鍵入的程式碼,
都可以立刻執行、得到回應。

執行 groovysh :

\begin{Verbatim}[commandchars=\\\{\}]
\PYG{n}{groovysh}
\end{Verbatim}

提示訊息如下:

\begin{Verbatim}[commandchars=@\[\]]
Groovy Shell (1.8.2, JVM: 1.6.0@_26)
Type 'help' or '\h' for help.
-------------------------------------------------------------------------------
groovy:000@textgreater[]
\end{Verbatim}

輸入一行 new Date() 指令,
按下 Enter 後可以得到以下輸出(顯示系統日期):

\begin{Verbatim}[commandchars=@\[\]]
groovy:000@textgreater[] new Date()
===@textgreater[] Tue Oct 04 16:34:40 CST 2011
\end{Verbatim}

接著再輸入一行,
又會立刻得到另一個執行結果。

\begin{Verbatim}[commandchars=@\[\]]
groovy:000@textgreater[] new Date().format('yyyy-MM-dd')
===@textgreater[] 2011-10-04
\end{Verbatim}

這種交談式的互動執行方式,
稱為 REPL (Read–eval–print loop)。
通常我們並不會用這種方式開發軟體,
但是需要測試某些程式片段或語法時,
這是一種相當簡便的用法。

舉例來說,
java.lang.String (字串類別)有許多「轉型」方法,
我們可能不記得有哪些方法名稱可以使用,
通常需要查閱 Java API 文件,
或是利用編輯器的自動完成(auto complete)功能。
但是在 GroovyShell 中,
我們只要輸入 String.to,
再按 \textbf{Tab} 鍵,
GroovyShell 就會幫我們找出 String 類別中,
以 to 開頭命名的所以方法。

\begin{Verbatim}[commandchars=@\[\]]
groovy:000@textgreater[] String.to

toBigDecimal()   toBigInteger()   toBoolean()      toCharacter()
toDouble()       toFloat()        toInteger()      toList()
toLong()         toSet()          toShort()        toString()
toURI()          toURL()          tokenize(        tokenize()
\end{Verbatim}

我們可以使用 GroovyShell 輕鬆解決以下的任務:
\begin{enumerate}
\item {} 
快速測試一小段 Java 程式是否能順利執行

\item {} 
測試及探索某個 Java API 的使用方法

\item {} 
接手維護一個舊有的 Java 函式庫,
原始碼和文件已經不見,
使用 Tab 鍵即可快速找出某個類別有哪些 public methods 。

\end{enumerate}


\subsection{Groovy compiler}
\label{intro:groovy-compiler}
使用 Groovy 撰寫的類別,
可以編譯成 Java byte code 、在 JVM 中執行。

建立一個命名為 Hello.groovy 的檔案,其程式碼內容如下。

\begin{Verbatim}[commandchars=\\\{\}]
\PYG{n}{println} \PYG{l+s+s1}{'Hello, World!'}
\end{Verbatim}

編譯成 byte code (用法類似 javac Hello.java):

\begin{Verbatim}[commandchars=@\[\]]
groovyc Hello.groovy
\end{Verbatim}

編譯成功後將會產生 \textbf{Hello.class} 檔案,
執行時不需要副檔名(用法類似 java Hello)。

\begin{Verbatim}[commandchars=@\[\]]
groovy Hello
\end{Verbatim}

因為 Groovy 程式碼可以透過 groovyc 編譯成 .class 檔案,
所以使用 Groovy 建立的函式庫,
也可以用 Java 程式語言存取。


\subsection{Groovy Script}
\label{intro:groovy-script}

\subsection{執行本書的範例程式碼}
\label{intro:id4}
本書的範例程式碼在 GitHub 的網址是:

\begin{Verbatim}[commandchars=@\[\]]
https://github.com/lyhcode/GroovyCookbook
\end{Verbatim}

您可以使用 git 指令下載:

\begin{Verbatim}[commandchars=@\[\]]
git clone git://github.com/lyhcode/GroovyCookbook.git
\end{Verbatim}

或是使用 GitHub 提供的 Downloads 按鈕,下載壓縮檔:

\begin{Verbatim}[commandchars=@\[\]]
https://github.com/lyhcode/GroovyCookbook/zipball/master
\end{Verbatim}

src 資料夾包含本書所有範例程式碼。

例如:
\begin{quote}

groovy src/HelloWorld.groovy
\end{quote}

如果程式執行正確,輸出如下:
\begin{quote}

Hello, World!
\end{quote}

本書的範例程式碼採用 Groovy 1.8.2 ,並在以下作業系統環境中測試:
\begin{itemize}
\item {} 
Ubuntu Linux 11.04 (Natty)

\item {} 
Mac OS X 10.6 (Snow Leopard)

\end{itemize}

如果在您的系統執行出現問題,煩請將問題回報給作者。


\chapter{RabbitMQ}
\label{rabbitmq::doc}\label{rabbitmq:rabbitmq}

\section{安裝 RabbitMQ}
\label{rabbitmq:id1}

\subsection{Ubuntu Linux}
\label{rabbitmq:ubuntu-linux}
使用 apt 安裝:

\begin{Verbatim}[commandchars=@\[\]]
sudo apt-get update
sudo apt-get install rabbitmq-server
\end{Verbatim}

通常在安裝後,系統就會自動啟動 RabbitMQ 服務,但您也可以手動以指令啟動:

\begin{Verbatim}[commandchars=@\[\]]
sudo service rabbitmq-server start
\end{Verbatim}

或重新啟動、停止:

\begin{Verbatim}[commandchars=@\[\]]
sudo service rabbitmq-server restart
sudo service rabbitmq-server stop
\end{Verbatim}


\subsection{Mac OS X}
\label{rabbitmq:mac-os-x}
使用 port 安裝:

\begin{Verbatim}[commandchars=@\[\]]
sudo port -d sync
sudo port install rabbitmq-server
\end{Verbatim}

啟動:

\begin{Verbatim}[commandchars=@\[\]]
sudo rabbitmq-server @&
\end{Verbatim}


\section{Hello, RabbitMQ}
\label{rabbitmq:hello-rabbitmq}
RabbitMQ 提供一組 Java AMQP client \footnote{
\href{http://www.rabbitmq.com/java-client.html}{http://www.rabbitmq.com/java-client.html}
} 函式庫,
在 Groovy 程式碼中,
只要以 Grape 的 @Grab 標記宣告即可使用。

\begin{Verbatim}[commandchars=@\[\]]
@PYGZat[]Grab('com.rabbitmq:amqp-client:2.6.1')
\end{Verbatim}

以下是「Hello, RabbitMQ」範例程式碼,展示簡易的 Sending / Receiving 動作如何完成。

\begin{Verbatim}[commandchars=\\\{\}]
\PYG{c+cm}{/*}
\PYG{c+cm}{ * src/rabbitmq/hello.groovy}
\PYG{c+cm}{ */}

\PYG{n+nd}{@Grab}\PYG{o}{(}\PYG{l+s+s1}{'com.rabbitmq:amqp-client:2.6.1'}\PYG{o}{)}

\PYG{k+kn}{import} \PYG{n+nn}{com.rabbitmq.client.*}

\PYG{n}{factory} \PYG{o}{=} \PYG{k}{new} \PYG{n}{ConnectionFactory}\PYG{o}{(}\PYG{o}{)}
\PYG{n}{factory}\PYG{o}{.}\PYG{n+na}{host} \PYG{o}{=} \PYG{l+s+s1}{'localhost'}
\PYG{n}{connection} \PYG{o}{=} \PYG{n}{factory}\PYG{o}{.}\PYG{n+na}{newConnection}\PYG{o}{(}\PYG{o}{)}
\PYG{n}{channel} \PYG{o}{=} \PYG{n}{connection}\PYG{o}{.}\PYG{n+na}{createChannel}\PYG{o}{(}\PYG{o}{)}

\PYG{n}{channel}\PYG{o}{.}\PYG{n+na}{queueDeclare}\PYG{o}{(}\PYG{l+s+s1}{'HELLO\PYGZus{}QUEUE'}\PYG{o}{,} \PYG{k+kc}{false}\PYG{o}{,} \PYG{k+kc}{false}\PYG{o}{,} \PYG{k+kc}{false}\PYG{o}{,} \PYG{k+kc}{null}\PYG{o}{)}

\PYG{c+c1}{//Sending}
\PYG{n}{message} \PYG{o}{=} \PYG{l+s+s1}{'Hello, RabbitMQ!'}
\PYG{n}{channel}\PYG{o}{.}\PYG{n+na}{basicPublish}\PYG{o}{(}\PYG{l+s+s1}{''}\PYG{o}{,} \PYG{l+s+s1}{'HELLO\PYGZus{}QUEUE'}\PYG{o}{,} \PYG{k+kc}{null}\PYG{o}{,} \PYG{n}{message}\PYG{o}{.}\PYG{n+na}{bytes}\PYG{o}{)}
\PYG{n}{println}\PYG{o}{(}\PYG{l+s+s2}{" [x] Sent '\PYGZdl{}message'"}\PYG{o}{)}

\PYG{c+c1}{//Receiving}
\PYG{n}{consumer} \PYG{o}{=} \PYG{k}{new} \PYG{n}{QueueingConsumer}\PYG{o}{(}\PYG{n}{channel}\PYG{o}{)}
\PYG{n}{channel}\PYG{o}{.}\PYG{n+na}{basicConsume}\PYG{o}{(}\PYG{l+s+s1}{'HELLO\PYGZus{}QUEUE'}\PYG{o}{,} \PYG{k+kc}{true}\PYG{o}{,} \PYG{n}{consumer}\PYG{o}{)}
\PYG{n}{delivery} \PYG{o}{=} \PYG{n}{consumer}\PYG{o}{.}\PYG{n+na}{nextDelivery}\PYG{o}{(}\PYG{o}{)}
\PYG{n}{message} \PYG{o}{=} \PYG{k}{new} \PYG{n}{String}\PYG{o}{(}\PYG{n}{delivery}\PYG{o}{.}\PYG{n+na}{body}\PYG{o}{)}
\PYG{n}{println}\PYG{o}{(}\PYG{l+s+s2}{" [x] Received '\PYGZdl{}message'"}\PYG{o}{)}

\PYG{n}{channel}\PYG{o}{.}\PYG{n+na}{close}\PYG{o}{(}\PYG{o}{)}
\PYG{n}{connection}\PYG{o}{.}\PYG{n+na}{close}\PYG{o}{(}\PYG{o}{)}
\end{Verbatim}


\chapter{Continuous Publishing}
\label{README:continuous-publishing}\label{README::doc}
Book Template for ContPub (Continuous Publishing).

這本書是「ContPub」的 GitHub 範本書,
您可以利用這個範本為基礎,開始製作您個人出版的書籍。

您可以在 GitHub 網址:

\begin{Verbatim}[commandchars=@\[\]]
https://github.com/contpub/BookTemplate
\end{Verbatim}

取得所有本電子書的原始碼。

如果您想預覽此範本產生的電子書,請 \href{http://goo.gl/XHCeQ}{下載} 以下 PDF 檔案。

\begin{Verbatim}[commandchars=@\[\]]
http://goo.gl/XHCeQ
\end{Verbatim}

您也可以利用 Google Docs Viewer 線上 \href{http://goo.gl/g4T8p}{預覽} 此書:

\begin{Verbatim}[commandchars=@\[\]]
http://goo.gl/g4T8p
\end{Verbatim}


\section{現在起,人人都能出版!}
\label{README:id3}
歡迎您加入開放電子書創作行列。

Continuous Publishing 是一項電子書出版改革,
ContPub 的理念就是讓每個人都能輕易利用電腦「書寫」,
並且在「毫不費力」的流程下將電子書出版發行。
如果您有興趣瞭解多一些,請參考以下資料:
\begin{itemize}
\item {} 
敏捷技術寫作(Agile Technical Writing)
\href{https://github.com/lyhcode/AgileTechnicalWriting}{https://github.com/lyhcode/AgileTechnicalWriting}

\item {} 
Lean Publishing
\href{http://leanpub.com/}{http://leanpub.com/}

\end{itemize}


\chapter{Groovy Taiwan}
\label{index:groovy-taiwan}
\href{http://www.facebook.com/groovy.taiwan/}{http://www.facebook.com/groovy.taiwan/}



\renewcommand{\indexname}{Index}
\printindex
\end{document}
